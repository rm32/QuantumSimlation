The algorithm is implemented using three classes: \emph{DeutschJoszaAlgorithm}, which contains the main algorithm set up; \emph{DeutschJoszaOracle}, which is the quantum gate representing (\( U_f \)) and contains the function; and \emph{DeutschJoszaOutput}, which contains the result, i.e. whether the function is constant or balanced.

When creating the oracle, the user can decide whether to create a constant or a balanced function. There are several balanced functions implemented, but the code needs to be changed to access them, as one balanced function is enough to demonstrate the algorithm.

The whole structure of the simulation is very close to the real implementation. Hadamard gates are applied to all qubits and then the \emph{DeutschJoszaOracle} gate (the transform \begin{math} U_f \end{math}) is applied to the whole register consisting of the \emph{n} input qubits and the additional output qubit. Only one \emph{QRegister} containing n+1 qubits is used.

The oracle gate uses direct bit manipulation and no matrix representation. The function \( f : \{0,1\}^n \to \{ 0,1 \} \) is separated from the apply-method (which represents \( U_f \) ), so it can be easily changed.